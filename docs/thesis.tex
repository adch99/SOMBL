\documentclass{article}
\usepackage{amsmath}
\usepackage{braket}
\usepackage{graphicx}
\usepackage{subcaption}
\usepackage[
    backend=biber,
    style=numeric]{biblatex}


\usepackage[]{hyperref}
\hypersetup{
    pdftitle={Thesis},
    pdfauthor={Aditya Chincholi},
    pdfpagemode=UseOutlines
}

\addbibresource{spin_orbit_coupled_mbl_zotero_citations.bib}

%opening
\title{Thesis}
\author{Aditya Chincholi (20181085)}

\begin{document}

\maketitle


% Context and Background
% - Introduce localization as a concept
% - Introduce Anderson Localization
%  - Anderson Localization in different dimensions
%  - Scaling Theory
%  - Weak Localization Effects
%  - Spin orbit coupling effects
%  - Weak Localization for spin-orbit coupling?
%  - Imbalance and Memory of Initial States

% My Work
% - Phase Space
% - Imbalance
% - Specialized Imbalances

\section{Localization}
We define localization of a particle at point $x$ as the wavefunction of a
particle that shows amplitudes which decay at least exponentially with the peak at $x$. 
The main question is that if we start with an initial state where the
particle is present at $x$, will the wavefunction remain localized at the point $x$
at long times?

\subsection{Anderson Localization}
The Anderson model, introduced by P. W. Anderson, is given by

\begin{align}
    H = \sum_{i} \epsilon_i c_{i}^{\dagger} c_{i} - t\sum_{\langle ij \rangle} c_{i}^{\dagger} c_{j}
\end{align}

where $t$ is the hopping strength, $\epsilon_i$ are random onsite potentials (known as disorder)
drawn from a uniform distribution $[-W/2,W/2]$ and $W$ is called the disorder strength.

In 1D, the model shows localization at any non-zero disorder strength $W$.


\subsection{Weak Localization}
In 2D, the scaling theory of localization requires a correction term to predict the resultant effect
as $d = 2$ forms the lower critical dimension for the transition. In the 2D Anderson model,
the phenomenon of Weak Localization results in the presence of localization even at arbitrarily low
disorders. However, the localization length in these cases can be extremely large and can easily
exceed the system size in finite size systems.

Weak localization is a constructive interference effect that increases the probability of
a particle to return to it's original site. 


\end{document}
