\documentclass{article}
\usepackage{amsmath}
\usepackage{braket}
\usepackage{graphicx}
\usepackage{subcaption}
\usepackage[
    backend=biber,
    style=numeric]{biblatex}


\usepackage[]{hyperref}
\hypersetup{
    pdftitle={Thesis},
    pdfauthor={Aditya Chincholi},
    pdfpagemode=UseOutlines
}

\addbibresource{spin_orbit_coupled_mbl_zotero_citations.bib}

%opening
\title{Thesis}
\author{Aditya Chincholi (20181085)}

\begin{document}

\maketitle


% Context and Background
% - Introduce localization as a concept
% - Introduce Anderson Localization
%  - Anderson Localization in different dimensions
%  - Scaling Theory
%  - Weak Localization Effects
%  - Spin orbit coupling effects
%  - Weak Localization for spin-orbit coupling?
%  - Imbalance and Memory of Initial States

% My Work
% - Phase Space
% - Imbalance
% - Specialized Imbalances

\section{Localization}
We define localization of a particle at point $x$ as the wavefunction of a
particle that shows amplitudes which decay at least exponentially with the peak at $x$. 
The main question is that if we start with an initial state where the
particle is present at $x$, will the wavefunction remain localized at the point $x$
at long times?

\subsection{Anderson Localization}
The following system was introduced by P. W. Anderson and 

\end{document}
