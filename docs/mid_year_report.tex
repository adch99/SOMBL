\documentclass[twocolumn]{article}
\usepackage{amsmath}
\usepackage{braket}
\usepackage{graphicx}
\usepackage{subcaption}
\usepackage[
    backend=biber,
    style=numeric]{biblatex}


\usepackage[]{hyperref}
\hypersetup{
    pdftitle={Localization in Spin-orbit Coupled Disordered Systems: Mid Year Report},
    pdfauthor={Aditya Chincholi},
    pdfpagemode=UseOutlines
}

\addbibresource{spin_orbit_coupled_mbl_zotero_citations.bib}

%opening
\title{Localization in Spin-orbit Coupled Disordered Systems: Mid Year Report}
\author{Aditya Chincholi (20181085)}

\begin{document}

\maketitle

\section{Introduction}

The 2D Anderson model with onsite disorder in the
tight-binding hamiltonian has been studied extensively.
However, the model does not account for spin related
effects. Our goal in this project is to explore the
localization trends for spin systems with spin-orbit
coupling. The addition of SOC (spin-orbit coupling) breaks
the spin symmetry of the system whilst preserving time
reversal invariance. We now have two quantities that can
potentially show interesting behaviour - charge/particle
density and spin density. Evangelou et al
\cite{evangelouAndersonTransitionTwo1987} showed the
existence of a localized-delocalized transition that is
absent in 2D tight-binding systems without spin-orbit
coupling. The transition involves spin systems with time
reversal invariance and hence, falls in the symplectic
universality class. The critical exponents for this
transition have been calculated to high accuracy, however, a
full analysis of the phase space and the mobility edges is
not present in literature
\cite{asadaAndersonTransitionTwoDimensional2002}. Recently,
experimental techniques in cold atom setups have made it
possible to create spin-orbit coupled tight-binding lattices
in the lab with tunable spin-orbit coupling strengths
\cite{orsoAndersonTransitionCold2017,huangExperimentalRealizationTwodimensional2016}.
Disordered systems without spin have been shown to retain
memory of their initial states in the form of persistant
imbalance in the particle density
\cite{chakrabortyMemoriesInitialStates2020}. The addition of
spin as a degree of freedom allows us to ask the same
question about information retention for the localized
states of the spin-orbit coupled disordered systems. Since
there are now two densities involved - charge density and
spin density, the two may show different localization
behaviours. Therefore, we aim to understand the phase space
and the long time imbalance dynamics of the system in more
detail in this project. 

\subsection{Aims}
\begin{enumerate}
    \item We want to understand the phase space of the
        system with respect to the three parameters:
        spin-orbit coupling strength $\alpha$, disorder
        strength $W$ and energy $E$ (we take $t = 1$ as a
        reference).

    \item We want to know if the localized wavefunctions
        correlate with any spin patterns.

    \item We want to know if the imbalance in charge density
        and spin density preserves memory of initial states
        similar to the spinless case. Is there a dynamical
        relation between the charge and spin density
        imbalance?
\end{enumerate}

The project seeks to answer all these questions, but has
only been able to answer the first one as of yet.

\begin{figure}[h]
    \centering
    \includegraphics[width=\linewidth]
    {../plots/PDFs/loc_lens_disorder_vs_coupling_heatmap.pdf}
    \caption{Heat map of mean localization lengths for $W \in [8,18]$,
        $\alpha \in [0,2]$ and for the long time limit on a 40x40 lattice.}
    \label{fig:meanloclenheatmap}
\end{figure}
\begin{figure}[h]
    \centering
    \includegraphics[width=\linewidth]
    {../plots/PDFs/loc_lens_disorder_vs_coupling_heatmap_upup.pdf}
    \caption{Heat map of $\uparrow\uparrow$ localization lengths for $W \in [8,18]$,
        $\alpha \in [0,2]$ and for the long time limit on a 40x40 lattice.}
    \label{fig:upupheatmap}
\end{figure}
\begin{figure}[h]
    \centering
    \includegraphics[width=\linewidth]
    {../plots/PDFs/loc_lens_disorder_vs_coupling_heatmap_updn.pdf}
    \caption{Heat map of $\uparrow\downarrow$ localization lengths for $W \in [8,18]$,
        $\alpha \in [0,2]$ and for the long time limit on a 40x40 lattice.}
    \label{fig:updnheatmap}
\end{figure}

\section{Spin-Orbit Coupled 2D Anderson Model}
\begin{align*}
    H = \sum_{i} \epsilon_i c_i^{\dagger} c_i
        - t\sum_{<ij>} c_i^{\dagger} c_j + H_{so}
\end{align*}
where

\begin{align*}
    H_{so} = \sum_{i}(&-\alpha c_{i_{x+1}\uparrow}^{\dagger}c_{i\downarrow}
            +\alpha c_{i_{x+1}\downarrow}^{\dagger}c_{i\uparrow} \\
    &+\alpha c_{i_{x-1}\uparrow}^{\dagger}c_{i\downarrow}
            -\alpha c_{i_{x-1}\downarrow}^{\dagger}c_{i\uparrow} \\
    &+ i\alpha c_{i_{y+1}\uparrow}^{\dagger}c_{i\downarrow}
            +i\alpha c_{i_{y+1}\downarrow}^{\dagger}c_{i\uparrow} \\
    &- i\alpha c_{i_{y-1}\uparrow}^{\dagger}c_{i\downarrow}
            -i\alpha c_{i_{y-1}\downarrow}^{\dagger} c_{i\uparrow})
\end{align*}

and $\epsilon_i \in [-W/2,W/2]$ are distributed uniformly
and we set $t = 1$. We use a Rashba type spin-orbit coupling
with a constant spin-orbit strength \cite{liuAndersonLocalizationDegenerate2016}.

When $\alpha = 0$, the model reduces to the 2D tight-binding
model and shows localization for any non-zero disorder
strength $W$ \cite{abrahamsScalingTheoryLocalization1979}.
However, the localization length $\xi$ can be very large and
therefore, localization may not be observed in systems with
small sizes. The primary contribution to localization comes
from the weak localization effect. The amplitudes for a
particle moving in a loop and returning to its original site
is enhanced because the amplitude of any closed path
constructively interferes with the the contribution of the
time-reversed path. However, this does not hold true when
spins are presented. In fact, the time-reversed path
amplitude now destructively interferes with the closed loop
path amplitude. 

We also note that the system is made up of fermions and has
time reversal invariance. Therefore, the spectrum consists
of paired energy levels due to Kramer's degeneracy. This may
be removed by adding a small magnetic field. We, however, do
not do this.

Along the lines of the metal-insulator transition in 3D, we
expect a mobility edge in the spectrum such that for $E >
E_c$, the system shows diffusive behaviour and for $E <
E_c$, the system shows localization. However, the system may
have multiple mobility edges which have not been probed yet.

\subsection{Methodology}
We use exact diagonalization to get eigenvectors of the hamiltonian
and create the long time green's function squared matrix.
\begin{enumerate}
    \item We create the hamiltonian $H$ on a $40\times40$
        lattice with open boundary conditions and diagonalize
        the hamiltonian using MKL/LAPACK.

    \item Using the eigenvectors we construct the long time
        retarded green's function matrix $|G_R|^2(i,j)$ given by
        the following expression \cite{chakrabortyMemoriesInitialStates2020}:
        \begin{align*}
        \langle|G_R^\infty|^2_{ij}\rangle_{dis} &=
            \langle\sum_{n}|\psi_n(i)^*\psi_n(j)|^2 \\
            &+ \sum_{\substack{m,n,\\E_m = E_n}}\psi_n(i)^* \psi_n(j) 
            \psi_m(j)^* \psi_m(i)\rangle_{dis} \\
            &= T_{nondeg} + T_{deg}
        \end{align*}

    \item The first term comes from the standard
        non-degenerate case of the long time limit of $|G_R|^2$.
        The second term arises becauses there is a pairwise
        degeneracy at each value of energy due to Kramer's
        degeneracy. The average is over all disorder
        realizations. Overall, these sums are $O(L^6)$ where $L$
        is the side length of the lattice. They constitute a
        major part of the computational effort required. They
        can be considerably sped up by casting them as matrix
        multiplications, albeit at the cost of space complexity.
        Define $A_{ij} = |\psi_{ij}|^2$ and $B_{in} =
        \psi_{i,2n}^* \psi_{i,2n+1}$ where $i,j = 1$ to $2L^2$
        (they contain spin indices as well) and $n = 1$ to
        $L^2$. Noting that $\psi_n(i) = \psi_{in}$, we have

        \begin{align*}
            T_{nondeg} &= \sum_{n}|\psi_n(i)^*\psi_n(j)|^2 \\
            &= \sum_n A_{in}A_{jn} = (AA^T)_{ij}
        \end{align*}
    
        and since the degenerate eigenvectors occur in pairs
        with our routine
    
        \begin{align*}
            T_{deg} &= \sum_{n=1}^{L^2}
            \psi_{2n}(i)^* \psi_{2n}(j) \psi_{2n+1}(j)^* \psi_{2n+1}(i)
            + h.c. \\
            &= \sum_{n=1}^{L^2} B_{in} B_{jn}^* = (B B^\dagger)_{ij}
        \end{align*}
    
    \item Using the matrix $|G_R^\infty|^2_{ij}$, we
        construct the function $|G_R^\infty|^2(r, \sigma, \sigma')$
        by binning and averaging the matrix. Explicitly,
        $|G_R^\infty|^2(r, \sigma, \sigma')$ represents the average value of the
        matrix between sites separated by a distance $r$:
        \[
            \frac{1}{N_{r,\sigma,\sigma'}}\sum_{|r_i - r_j| = r}|G_R^\infty|^2_{ij}
        \]
        We compute the function for $r \in [0, L]$. $r$ can go
        up to $L\sqrt{2}$, however, the number of points per bin
        reduces drastically and would contribute more to the
        error. 
    
    \item For a localized function, $|G_R^\infty|^2(r, \sigma, \sigma')$
        varies as $e^{-2r/\xi}$. So we fit a line to the data
        $ln\left[|G_R^\infty(r,\sigma,\sigma')|^2\right]$ vs $r$ in order to
        obtain a localization length. Since we consider open
        boundary conditions, we can only reliably claim to
        measure upto $\xi_{\sigma,\sigma'} = L / 2$. For higher values of the
        fitted parameter, we can only say that the wavefunction
        is effectively delocalized for our lattice size. In this
        manner, we obtain 4 localization lengths: $\xi_{\uparrow\uparrow}$,
        $\xi_{\uparrow\downarrow}$, $\xi_{\downarrow\uparrow}$
        and $\xi_{\downarrow\downarrow}$. By virtue of symmetry of up
        and down spins, we have essentially two independent quantities as
        $\xi_{\uparrow\downarrow} = \xi_{\downarrow\uparrow}$ and
        $\xi_{\uparrow\uparrow} = \xi_{\downarrow\downarrow}$.
\end{enumerate}

% \onecolumn
\begin{figure}[h]
    \centering
    \includegraphics[width=\linewidth]
    {../plots/PDFs/mbl_40x40_W11_C1_TU1_TD1_N100_upup_distvsgfsq.pdf}
    \caption{$|G_{R,\uparrow\uparrow}|^2(r)$ vs $r$ at $\alpha=1.0$,
            $W=11$ on a semi-log scale for 40x40 lattice at long times.}
    \label{fig:genupupgfuncsq}
\end{figure}
\begin{figure}[h]
    \centering
    \includegraphics[width=\linewidth]
    {../plots/PDFs/mbl_40x40_W11_C1_TU1_TD1_N100_updn_distvsgfsq.pdf}
    \caption{$|G_{R,\uparrow\downarrow}|^2(r)$ vs $r$ at $\alpha=1.0$,
            $W=11$ on a semi-log scale for 40x40 lattice at long times.}
    \label{fig:genupdngfuncsq}
\end{figure}

\section{Results}
\subsection{Phase Space}
We calculated the localization length for different values of
$\alpha$ and $W$. The resultant mean localization lengths are shown
in Figure \ref{fig:meanloclenheatmap}. These values clearly
indicate the presence of a transition. The disorder acts to
increase localization while the spin-orbit coupling delocalizes
the system. However, the functions $|G_R^\infty|^2(r, \sigma, \sigma')$
exhibit more information than just this much. There are two things
of note here. First, there is a marked difference in the structure
of the functions $|G_{R,\uparrow\uparrow}^\infty|^2(r)$ and
$|G_{R,\uparrow\downarrow}^\infty|^2(r)$. The former peaks at
$r = 0$ whereas the latter does not always do that. 
Moreover, there are some artefacts near the $r = 0$ and $r = L$
ends of the functions. Slight oscillations also exist in Figure
\ref{fig:upupheatmap} and Figure \ref{fig:updnheatmap}. These
are expected to be boundary effects but this needs to be verified
by changing the fitting criterion.

However, the data and fits obtained are not particularly
good. Therefore, we cannot reliably conclude anything
without addressing the following concerns: 
\begin{enumerate}
    \item The presence of open boundary conditions and
        finite lattice sizes introduces the possibility of
        boundary effects and finite size effects which can cause
        unwanted artefacts in our wavefunctions. While one may
        think that periodic boundary conditions solve this
        issue, the periodic boundary conditions restrict us
        to half the system size and estimating the localization
        length on a torus is not well-defined when $\xi$ is
        comparable to the circumferences.

    \item Disorder averaging inherently introduces error in
        terms of averaging and this should be accounted for
        before drawing any serious conclusions from the data.
        For this reason, we average over $N = 100$ disorder
        realizations.
    
    \item Fitting an exponential to $|G_R^\infty|^2(r)$ is
        prone to errors and can lead to unreliable data since
        the function is not exponentially decaying for all
        values of $r$. In particular, the function is decays
        exponentially only for sufficiently large values of $r$,
        it doesn't have this behaviour near $r = 0$. But for
        higher values of $r$, we have lesser number of data
        points as the simulation has a finite size. Therefore,
        prudence is advised before drawing conclusions.    
\end{enumerate}

\onecolumn
\begin{figure}[h]
\begin{subfigure}[b]{0.5\textwidth}
    \centering
    \includegraphics[width=0.8\linewidth]
    {../plots/PDFs/mbl_40x40_W17_C0.4_TU1_TD1_N100_upup_distvsgfsq.pdf}
    \caption{$|G_{R,\uparrow\uparrow}|^2(r)$ vs $r$ at $\alpha=0.4$,
            $W=17$ on a semi-log scale for 40x40 lattice at long times.}
    \label{fig:locupupgfuncsq}
\end{subfigure}
\hfill
\begin{subfigure}[b]{0.5\textwidth}
    \centering
    \includegraphics[width=0.8\textwidth]
    {../plots/PDFs/mbl_40x40_W17_C0.4_TU1_TD1_N100_updn_distvsgfsq.pdf}
    \caption{$|G_{R,\uparrow\downarrow}|^2(r)$ vs $r$ at $\alpha=0.4$,
            $W=17$ on a semi-log scale for 40x40 lattice at long times.}
    \label{fig:locupdngfuncsq}
\end{subfigure}

\begin{subfigure}{0.5\textwidth}
    \centering
    \includegraphics[width=0.8\textwidth]
    {../plots/PDFs/mbl_40x40_W10_C1.5_TU1_TD1_N100_upup_distvsgfsq.pdf}
    \caption{$|G_{R,\uparrow\uparrow}|^2(r)$ vs $r$ at $\alpha=1.5$,
            $W=10$ on a semi-log scale for 40x40 lattice at long times.}
    \label{fig:delocupupgfuncsq}
\end{subfigure}
\hfill
\begin{subfigure}{0.5\textwidth}
    \centering
    \includegraphics[width=0.8\textwidth]
    {../plots/PDFs/mbl_40x40_W10_C1.5_TU1_TD1_N100_updn_distvsgfsq.pdf}
    \caption{$|G_{R,\uparrow\downarrow}|^2(r)$ vs $r$ at $\alpha=1.5$,
            $W=10$ on a semi-log scale for 40x40 lattice at long times.}
    \label{fig:delocupdngfuncsq}
\end{subfigure}

\begin{subfigure}{0.5\textwidth}
    \centering
    \includegraphics[width=0.8\textwidth]
    {../plots/PDFs/mbl_40x40_W9_C0.2_TU1_TD1_N100_upup_distvsgfsq.pdf}
    \caption{$|G_{R,\uparrow\uparrow}|^2(r)$ vs $r$ at $\alpha=0.2$,
            $W=9$ on a semi-log scale for 40x40 lattice at long times.}
    \label{fig:lowdisupupgfuncsq}
\end{subfigure}
\hfill
\begin{subfigure}{0.5\textwidth}
    \centering
    \includegraphics[width=0.8\textwidth]
    {../plots/PDFs/mbl_40x40_W9_C0.2_TU1_TD1_N100_updn_distvsgfsq.pdf}
    \caption{$|G_{R,\uparrow\downarrow}|^2(r)$ vs $r$ at $\alpha=0.2$,
            $W=9$ on a semi-log scale for 40x40 lattice at long times.}
    \label{fig:lowdisupdngfuncsq}
\end{subfigure}
\end{figure}
\twocolumn

\section{Conclusions}

The fitting issues and noisiness of the data indicate that
we cannot draw any solid conclusions from the data apart
from the overall trend. The big picture is that there is a
phase transition present in the system from localized to
delocalized phase. The effect of spin-orbit coupling is
to push the system towards delocalization while the disorder
strength correlates with increasing localization. Having seen
this picture, we can ask a plethora of new questions about
the system that we will attempt to answer in the rest of
the project, namely whether certain spin patterns get localized,
the structure of the spectrum and mobility edges and the dynamics
between the charge and spin density imbalances.

% \clearpage

\printbibliography

\end{document}